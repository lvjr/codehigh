%  -*- coding: utf-8 -*-
\documentclass[oneside]{book}
\usepackage[a4paper,margin=2cm]{geometry}

\newcommand*{\myversion}{2025B}
\newcommand*{\mydate}{Version \myversion\ (\the\year-\mylpad\month-\mylpad\day)}
\newcommand*{\mylpad}[1]{\ifnum#1<10 0\the#1\else\the#1\fi}

\setlength{\parindent}{0pt}
\setlength{\parskip}{4pt plus 1pt minus 1pt}

\usepackage{codehigh}
\usepackage{hyperref}
\hypersetup{
  colorlinks=true,
  urlcolor=blue3,
  linkcolor=green3,
}
\usepackage{tabularx,booktabs}

\begin{document}

\title{\textsf{\color{green3}Codehigh: Highlight Codes and Demos with l3RegEx and LPeg}}
\author{Jianrui Lyu (tolvjr@163.com)\\ \url{https://github.com/lvjr/codehigh}}
\date{\mydate}
\maketitle

\tableofcontents

\chapter{Package Interfaces}

\section{Introduction}

\verb!Codehigh! package uses \verb!l3regex!\footnote{\url{https://www.ctan.org/pkg/l3regex}}
package in \LaTeX3 Programming Layer to parse and highlight source codes and demos.
It is more powerful than \verb!listings! package, and more easy to use than \verb!minted! package.
But it is slower than both of them.
Therefore in LuaTeX the package provides another way to highlight code: using \verb!LPeg!%
\footnote{\url{http://www.inf.puc-rio.br/~roberto/lpeg/}}.
\verb!LPeg! is much more powerful and faster than \verb!l3regex!.

%At present, this package is in \underline{\color{red3}\textbf{experimental}} status.
%Don’t use it in important documents, unless you have time
%to update them for the newer versions of \verb!codehigh! package in the future.

\section{Highlighting Code}

There are several predefined languages: \verb!latex!, \verb!latex/latex2!, \verb!latex/latex3!,
\verb!latex/math! and \verb!latex/table!.
The following example is typeset by \verb!codehigh! environment with default option \verb!language=latex!.
\begin{codehigh}
%  -*- coding: utf-8 -*-
\documentclass{article}
\usepackage[a4paper,margin=2cm]{geometry}
\usepackage{codehigh}
\usepackage{hyperref}
\newcommand*{\myversion}{2021C}
\newcommand*{\mydate}{Version \myversion\ (\the\year-\mylpad\month-\mylpad\day)}
\newcommand*{\mylpad}[1]{\ifnum#1<10 0\the#1\else\the#1\fi}
\setlength{\abc}{1}
\begin{document}
% some comment
\section{Section Name}
\subsection*{Suction Name}
Math $a+b$.
\end{document}
\end{codehigh}

The following example is typeset by \verb!codehigh! environment with option \verb!language=latex/latex2!.
\begin{codehigh}[language=latex/latex2]
\def\abcd#1#2{
  % some comment
  \unskip
  \setlength{\parindent}{0pt}%
  \setlength{\parskip}{0pt}%
  \setcounter{choice}{0}%
  \let\item=\my@item@temp
  \settowidth{\my@item@len}{\vbox{\halign{##1\hfil\cr\BODY\crcr}}}%
  \setcounter{choice}{0}%
}
\end{codehigh}
This language is for highlighting LaTeX2 classes and packages.
It highlights private commands and public commands with different colors.

The following example is typeset by \verb!codehigh! environment with option \verb!language=latex/latex3!.
\begin{codehigh}[language=latex/latex3]
\cs_new_protected:Npn \__codehigh_typeset_demo:
  {
    \__codehigh_build_code:
    \__codehigh_build_demo:
    \dim_set:Nn \l_tmpa_dim { \box_wd:N \g__codehigh_code_box }
    \dim_set:Nn \l_tmpb_dim { \box_wd:N \g__codehigh_demo_box }
    \par\addvspace{0.5em}\noindent
    % more code
  }
\end{codehigh}
This language is for highlighting LaTeX3 classes and packages.
It highlights private commands/variables and public commands/variables with different colors.

The following example is typeset by \verb!codehigh! environment with option \verb!language=latex/math!.
\begin{codehigh}[language=latex/math]
\begin{align}
  \pi\left[\frac13z^3\right]\sin(2x+1)_0^4 = \frac{64}{3}\pi
\end{align}
\end{codehigh}

The following example is typeset by \verb!codehigh! environment with option \verb!language=latex/table!.
\begin{codehigh}[language=latex/table]
\begin{tabular}[b]{|lc|r|}
\hline
One   &  Two  & Three \\
%\hline
Four  & Five  &   Six \\
\hline%\hline\hline
Seven & Eight &  Nine \\
\hline
\end{tabular}
\end{codehigh}

\section{Highlighting Demo}

The followings are typeset by \verb!demohigh! environment with option \verb!language=latex/table!.
\begin{demohigh}[language=latex/table]
\begin{tabular}{lccr}
\hline
 Alpha   & Beta  & Gamma  & Delta \\
\hline
 Epsilon & Zeta  & Eta    & Theta \\
\hline
 Iota    & Kappa & Lambda & Mu    \\
\hline
\end{tabular}
\end{demohigh}
\begin{demohigh}[language=latex/table]
\begin{tabular}{llccrr}
\hline
 Alpha & Beta  & Gamma & Delta & Epsilon & Zeta \\
\hline
 Eta   & Theta & Iota  & Kappa & Lambda & Mu \\
\hline
\end{tabular}
\end{demohigh}
Note that \verb!demohigh! environment will measure the width of source lines.
When it is too large, the result will be put below.

\section{Highlighting File}

Using \verb!\dochighinput! command, you can input and highlight some file.
The last chapter of this manual is typeset with the following code line:
\begin{codehigh}
\dochighinput[language=latex/latex3]{codehigh.sty}
\end{codehigh}

In reading an input file, lines starting wtih \verb!%%%! will be omitted,
and lines starting with \verb!%%>! will be extracted and typeset as normal text.

\section{Customization}

The following example changes default background colors with \verb!\CodeHigh! command:
\begin{codehigh}
\CodeHigh{language=latex/table,style/main=yellow9,style/code=red9,style/demo=azure9}
\end{codehigh}
\CodeHigh{language=latex/table,style/main=yellow9,style/code=red9,style/demo=azure9}
Note that \verb!codehigh! package will load \verb!ninecolors!%
\footnote{\url{https://www.ctan.org/pkg/ninecolors}} package for proper color contrast.
\begin{demohigh}
\begin{tabular}{lccr}
\hline
 Alpha   & Beta  & Gamma  & Delta \\
\hline
 Epsilon & Zeta  & Eta    & Theta \\
\hline
 Iota    & Kappa & Lambda & Mu    \\
\hline
\end{tabular}
\end{demohigh}

To modify or add languages and themes, please read the source files
\verb!codehigh.sty! and \verb!codehigh.lua! for reference.

\section{Fake Verbatim Command}

\CodeHigh{language=latex/table,style/main=gray9,style/code=gray9,style/demo=white}

To ease the pain of writing verbatim commands
(such as in \verb|tabularx| and \verb|tabularray| tables),
This package provides \verb|\fakeverb| command.

This command will remove the backslashes in the following control symbols
before typesetting its content:

\renewcommand\arraystretch{1.3}
\begin{center}
\begin{tabularx}{0.9\linewidth}{llX}
\toprule
Input           & Result        & Remark \\
\midrule
\fakeverb{\\\\} & \fakeverb{\\} &
  Need to be escaped only when typesetting other control symbols in this table \\
\fakeverb{\\\{}	& \fakeverb{\{} &
  Need to be escaped only when left and right curly braces are unmatched \\
\fakeverb{\\\}}	& \fakeverb{\}} &
  Need to be escaped only when left and right curly braces are unmatched \\
\fakeverb{\\\#}	& \fakeverb{\#} &
  Always need to be escaped \\
\fakeverb{\\\^}	& \fakeverb{\^} &
  Need to be escaped only when there are more than one in a row \\
\texttt{\textbackslash\textvisiblespace} & \texttt{\textvisiblespace} &
  Need to be escaped only when more than one in a row or after control words \\
\fakeverb{\\\%}	& \fakeverb{\%} &
  Always need to be escaped \\
\bottomrule
\end{tabularx}
\end{center}

The argument of \verb|\fakeverb| command need to be enclosed with curly braces.
Therefore it could be safely used inside \verb|tabularray| tables and other LaTeX commands.

Here is an example of using \verb!\fakeverb! commands inside \verb|tabularx| environment:

\begin{demohigh}[language=latex/table]
\begin{tabularx}{0.5\textwidth}{lX}
\hline
 Alpha & \fakeverb{\abc{}$&^_^uvw 123} \\
\hline
 Beta  & \fakeverb{\bfseries\ \#\%} \\
\hline
\end{tabularx}
\end{demohigh}

Here is another example of using \verb!\fakeverb! commands inside \verb|\fbox| command:

\begin{demohigh}[language=latex/latex2]
Hello\fbox{\fakeverb{$\left\\\{A\right.$\#}}Verb!
\end{demohigh}

\chapter{The Source Code}

\dochighinput[language=latex/latex3]{codehigh.sty}

\end{document}
